% Adapted from Elchanan Mossel's template
\documentclass[11pt]{article}
\usepackage{amssymb}
\usepackage{amsfonts}
\usepackage{amsmath}
\usepackage{algorithm}
\usepackage[noend]{algpseudocode}
\usepackage{hyperref,graphicx}
\usepackage{mathtools}
\usepackage{bm}
\usepackage{lipsum}
\usepackage{latexsym}
\usepackage{multicol,caption}
\usepackage{tikz}
\usepackage{epsfig}
\usepackage{fancybox}
\setlength{\evensidemargin}{.25in}
\setlength{\textwidth}{6in}
\setlength{\topmargin}{-0.6in}
\setlength{\textheight}{8.5in}
\newenvironment{Figure}
  {\par\medskip\noindent\minipage{\linewidth}}
  {\endminipage\par\medskip}
 %\usepackage[sort&compress,square,comma,authoryear]{natbib}

\newcommand{\handout}[6]{
   \renewcommand{\thepage}{#1-\arabic{page}}
   \noindent
   \begin{center}
%\framebox{
\doublebox{
      \vbox{
   \hbox to 5.78in {{\sf Photoacoustic Tomography}
\hfill \sf #2 }
       \vspace{4mm}
       \hbox to 5.78in { {\Large \hfill #5  \hfill} }
       \vspace{2mm}
       \hbox to 5.78in { {\em #3 \hfill #4} }
      }
}

%}
   \end{center}
   \vspace*{4mm}
}

\newcommand{\lecture}[4]{\handout{}{#2}{Date: #3}{ #4}{Time Reversal Related Numerical Experiments}}

\textwidth=6in
\oddsidemargin=0.25in
\evensidemargin=0.25in
\topmargin=-0.1in
\footskip=0.8in
\parindent=0.0cm
\parskip=0.3cm
\textheight=8.00in
\setcounter{tocdepth} {3}
\setcounter{secnumdepth} {2}
\sloppy
\newtheorem{theorem}{Theorem}
\newtheorem{lemma}[theorem]{Lemma}
\newtheorem{proposition}[theorem]{Proposition}
\newtheorem{corollary}[theorem]{Corollary}
\newtheorem{fact}[theorem]{Fact}
\newtheorem{definition}[theorem]{Definition}
\newtheorem{remark}[theorem]{Remark}
\newtheorem{conjecture}[theorem]{Conjecture}
%\newtheorem{algorithm}[theorem]{Algorithm}
\newtheorem{question}[theorem]{Question}
\newtheorem{answer}[theorem]{Answer}
\newtheorem{exercise}[theorem]{Exercise}
\newtheorem{example}[theorem]{Example}
\newenvironment{proof}{\noindent \textbf{Proof:}}{$\Box$}

\newcommand{\ignore}[1]{}

\newcommand{\prob}{\mathbb{P}}
\newcommand{\cX}{\mathcal{X}}
\newcommand{\mY}{\mathbf{Y}}
\newcommand{\mX}{\mathbf{X}}
\newcommand{\my}{\mathbf{y}}
\newcommand{\mH}{\mathbf{H}}
\newcommand{\ob}{\hat{\beta}^{\text{ols}}}
\newcommand{\rb}{\hat{\beta}^{\text{ridge}}}
\newcommand{\lb}{\hat{\beta}^{\text{lasso}}}
\newcommand{\Tr}{\text{Tr}}

\newcommand{\Var}{\text{Var}}
\newcommand{\Cov}{{\bf Cov}}

\usepackage{mathtools}
\newcommand{\Inf}{\mathrm{Inf}}
\newcommand{\I}{\mathrm{I}}
\newcommand{\J}{\mathrm{J}}

\newcommand{\eps}{\epsilon}
\newcommand{\lam}{\lambda}
\newcommand{\N}{\mathbb N}
\newcommand{\R}{\mathbb R}
\newcommand{\Z}{\mathbb Z}
\newcommand{\C}{\mathbb C}
\newcommand{\CalE}{{\mathcal{E}}}

\newcommand{\CalU}{{\mathcal{U}}}
\newcommand{\F}{{\mathcal{F}}}
\newcommand{\boldsigma}{{\boldsymbol \sigma}}
\newcommand{\boldupsilon}{{\boldsymbol \upsilon}}
\def\alert#1{\textcolor{red}{#1}}
\newcommand{\qed}{\ \ \rule{1ex}{1ex}}
\newcommand{\To}{\longrightarrow}
\def\ind{{\rm 1\hspace{-0.
0ex}1}}
\renewcommand{\labelitemii}{$\star$}
\newcommand{\cP}{\mathcal{P}}
\newcommand{\cH}{\mathcal{H}}
\newcommand{\V}{\mathcal{V}}
\newcommand{\bX}{\mathbb{X}}
\newcommand{\by}{\mathcal{\pmb{y}}}
\newcommand{\sgn}{\text{sgn}}
\newcommand{\tk}{\tilde{k}}
\newcommand{\bphi}{\bar{\phi}}
\usepackage{color}
\newcommand{\E}{\mathbb{E}}
\DeclareMathOperator*{\argmin}{arg\,min}
\DeclareMathOperator*{\argmax}{arg\,max}
\newcommand{\B}{\mathcal{B}}
\newcommand{\HH}{\mathcal{H}}
\newcommand{\Lb}{\mathbf{L}}


%\setcounter{section}{5}
\begin{document}
\lecture{Time Reversal Numerical Experiments}{Spring 2020}{\today}{Shuyang Qin}

\section{Forward Modeling}
Throughout the whole experimental process, I used the finite difference scheme to simulate wave propagation in unbounded domain. The scheme follows the paper: '\textbf{The perfectly matched layer for acoustic waves in absorptive media}' by Qing-Huo Liu and Jianping Tao. To be more specific, the original second order wave equation is decomposed into 2 first order equations:
\begin{equation*}
\begin{cases}
\partial_{t} \mathbf{v}(\mathbf{x},t) = -\nabla p(\mathbf{x},t),\\
\partial_t p(\mathbf{x},t) = -c^2(\mathbf{x}) \nabla \cdot \mathbf{v}(\mathbf{x},t).
\end{cases}
\end{equation*}
Here $p(\mathbf{x},t)$ denotes pressure field, and vector $\mathbf{v}(\mathbf{x},t)$ denotes particle velocity fields.

By complex coordinate stretching, the above formula in 2D can be modified as
\begin{equation*}
\begin{cases}
\partial_t \mathbf{v}_j(\mathbf{x},t) + \omega_j(\mathbf{x}) \mathbf{v}_j(\mathbf{x},t) = -\partial_j p(\mathbf{x},t), \ \  \textup{for} \ j = 1,2,\\
\partial_t p_j(\mathbf{x},t) + \omega_j(\mathbf{x}) p_j(\mathbf{x},t) = -c^2(\mathbf{x}) \partial_j \mathbf{v}_j(\mathbf{x},t), \ \  \textup{for} \ j = 1,2,
\end{cases}
\end{equation*}
where $\mathbf{x} = (x_1,x_2), \ \mathbf{v} = (\mathbf{v}_1,\mathbf{v}_2)$, $\partial_j$ denotes partial derivative along $x_j$, and $p = p_1 + p_2$. Also, $\omega_j$ denotes the absorption coefficient along axis $x_j$ in computational domain. According to kWave package, the absorption coefficient is chosen to be a fourth order polynomial:
\begin{equation*}
\omega_j(\mathbf{x}) =
\begin{cases}
 \textup{const}*\Big(\frac{\textup{dist}(\mathbf{x},\textup{PML})}{\textup{length of PML}}\Big)^4\\
 0, \ \ \ \ \ \textup{otherwise}
 \end{cases}
\end{equation*}
And the forward modeling follows standard-staggered finite difference scheme. We compute the pressure field at uniform standard grids $p(x_j,y_k,t_n)$, and particle velocity field at staggered grid $v_x(x_{j+1/2},y_k,t_{n+1/2})$, $v_y(x_j,y_{k+1/2},t_{n+1/2})$. The second order finite difference scheme becomes
\begin{equation*}
111
\end{equation*}



\end{document}